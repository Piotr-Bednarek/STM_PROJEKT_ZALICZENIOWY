\documentclass[12pt, a4paper]{article}

% Pakiety kodowania i językowe
\usepackage[utf8]{inputenc}
\usepackage[T1]{fontenc}
\usepackage[polish]{babel}

% Pakiety do geometrii i 
\usepackage{geometry}
\usepackage{graphicx}
\usepackage{float}
\usepackage{hyperref}
\usepackage{listings}
\usepackage{xcolor}

% Konfiguracja geometrii strony
\geometry{
    a4paper,
    total={170mm,257mm},
    left=25mm,
    top=25mm,
    bottom=25mm,
    right=25mm
}

% Konfiguracja linków
\hypersetup{
    colorlinks=true,
    linkcolor=black,
    filecolor=magenta,      
    urlcolor=blue,
}

% Konfiguracja wyświetlania kodu
\lstset{
    basicstyle=\ttfamily\footnotesize,
    breaklines=true,
    captionpos=b,
    frame=single,
    numbers=left,
    numberstyle=\tiny,
    keywordstyle=\color{blue},
    commentstyle=\color{green!60!black},
    stringstyle=\color{purple}
}

% Dane dokumentu
\title{Układ sterowania położeniem piłki pingpongowej na belce}
\author{Piotr Bednarek \\ Jan Andrzejewski \\ Mateusz Banaszak}
\date{\today}

\begin{document}

\maketitle

\section{Wstęp}
Przedmiotem projektu jest układ automatycznej regulacji pozycji piłki pingpongowej na
pochylonej belce. System wykorzystuje serwo TowerPro MG946R do sterowania kątem nachylenia belki
oraz laserowy czujnik odległości Time-of-Flight VL53L0X do pomiaru aktualnego położenia
piłki. Mikrokontroler STM32F767ZI analizuje dane z czujnika i w czasie rzeczywistym
koryguje nachylenie belki, aby utrzymać piłkę w zadanej pozycji lub śledzić określoną
trajektorię. Układ stanowi klasyczny przykład zastosowania regulacji PID w systemie
niestabilnym, gdzie niewielkie zakłócenia mogą prowadzić do utraty kontroli nad obiektem.

\section{Opis sprzętu}
Opis wykorzystanego mikrokontrolera (STM32) oraz peryferiów (czujniki, moduły).

\begin{itemize}
    \item STM Nucleo F767ZI
    \item Servo TowerProMG90s
    \item Grove - VL53LOX Timeof Flight I2C
\end{itemize}

\section{Opis oprogramowania}
\subsection{Struktura projektu}
Opis struktury plików i katalogów.

\subsection{Kluczowe algorytmy}
Opis najważniejszych fragmentów kodu.

\section{Testy i wyniki}
Opis przeprowadzonych testów i otrzymanych wyników.

\section{Podsumowanie}
Wnioski końcowe i możliwości rozwoju projektu.

\end{document}
